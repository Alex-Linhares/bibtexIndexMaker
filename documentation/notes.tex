\documentclass[a4paper,oneside]{article}
\usepackage{geometry}
\usepackage[latin1]{inputenc}
\usepackage[catalan]{babel}
\usepackage{amsfonts}
\usepackage{url} 

\usepackage[pdfauthor={Ramon Xuriguera Albareda},%
		pdfsubject={BibTeX Bibliography Index Maker},%
		pdftitle={BibTeX Bibliography Index Maker: Notes},%
		pdftex]{hyperref}
		

\title{BibTeX Bibliography Index Maker: Notes}
\author{Ramon Xuriguera}
\date{}

\begin{document}
\maketitle

\section{BibTeX}
Aspectes del format BibTeX a tenir en compte:
\begin{itemize}
\item{Com podem distingir entre diferents tipus d'entrada (article, book, inproceedings, etc.) a partir del fitxer?}
\item{Format dels noms. Un nom consisteix de diferents parts: First, von, Last, Jr. El token \textit{von} o \textit{de la} cal posar-los
en min�scules. Per tal que el nom es reconegui, cal que tingui el format: von Last, Jr, First. D'aquesta manera, si hi ha m�s d'un
cognom no passa res.}
\item{Car�cters Unicode entre claus per poder ser utilitzats correctament amb l'estil \textit{alpha}. Per exemple: \verb=Jos{\'{e}}=}
\item{Per prevenir que BibTeX canvi� un text a min�scules, cal posar el text entre claus.}
\item{Si hi ha massa autors, truncar la llista amb \textit{et al.}}
\item{Utilitzar abreviatures de tres lletres per als mesos}
\item{Utilitzar el camp \texttt{key} per a organitzacions amb un nom llarg, de manera que s'utilitzin les inicials de l'organitzaci� 
al fer una cita.}

\end{itemize}

\section{PDF}
Aspectes a considerar:
\begin{itemize}
\item{Car�cters Unicode}
\item{Glyphs com ara \textit{fi} corresponen a m�s d'una lletra}
\item{Les llibreries extreuen el text per l�nies}
\item{Podem obtenir les seg�ents dades del fitxer sense
haver-ne d'extreure el text: n�mero de p�gines, t�tol, autor, assumpte i paraules clau. Si aquests camps no s'han omplert al 
generar el fitxer, estaran en blanc.}
\end{itemize}

\subsection{Text extraction}
\subsubsection{PDFBox}
Llibreria escrita en Java i publicada sota la llic�ncia \textit{Apache License v2.0}. Actualment es troba a la incubadora d'Apache. 
\subsubsection{PDFMiner}
Programes desenvolupats en Python per extreure i analitzar dades de documents PDF.
\subsubsection{jPod}

\section{Articles Db}
\begin{description}
\item{\textbf{DBLP}}\\
DBLP proporciona una API molt simple per cercar autors autors i obtenir les seves publicacions, per� no al rev�s. Al cercar per autor, es mostren
les claus de cada publicaci�, per� no el t�tol.
\item{\textbf{Portal ACM}}
\item{\textbf{Google Scholar}}\\
Enlla� directe a la cita en format BibTeX. Actualment no proporcionen cap API per obtenir-ho. Caldria fer-ho amb \texttt{wget}.
\item{\textbf{CiteSeerX}}
\item{\textbf{Arxiv.org}}\\
	\url{http://arxiv.org/help/api/index}
\end{description}

\section{Software existent}
\url{http://en.wikipedia.org/wiki/Comparison_of_reference_management_software}
\begin{description}
\item{\textbf{JabRef}}\\
Java, llic�ncia: GPL.
\\
�s possible crear plug-ins amb \textit{Java Plug-in Framework}.\\
\url{http://jabref.sourceforge.net/}

\item{\textbf{Mendeley}}\\
Propietari, no accepta plug-ins.\\
\url{http://www.mendeley.com/}

\item{\textbf{BibDesk}}\\
Objective C, BSD. \\
\url{http://bibdesk.sourceforge.net/}
\end{description}

\section{Llenguatge}
\begin{itemize}
\item{Java}
\item{Python}
\item{Java + Python amb Jython}
\item{C\# i IronPython}
\end{itemize}
\end{document}