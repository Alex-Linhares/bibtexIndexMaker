\chapter{Biblioteques utilitzades}
Per una banda fen servir els seg�ents m�duls de la biblioteca est�ndard. No s�n els �nics que utilitzem, per� s� els que ofereixen funcionalitats m�s interessants:
\begin{itemize}
\item{RE:}
Ofereix la possibilitat de treballar amb expressions regulars i amb una sintaxi molt rica. L'hem fet servir �mpliament a tot el sistema.

\item{SimpleJSON:}
L'hem utilitzat per dos objectius diferents: llegir fitxers en format JSON i per serialitzar i desserialitzar objectes \textit{Python} a l'emmagatzemar-los a la base de dades. Els objectes serialitzats utilitzant aquest m�dul tenen la caracter�stica que s�n llegibles per les persones. Per exemple, el resultat de serialitzar un diccionari �s: \texttt{'\{``clau01'': 4, ``clau03'': 15, ``clau02'': 8\}'}

\item{DiffLib:}
Ens permet fer \textit{string matching} i obtenir llistes de blocs coincidents aix� com un �ndex de similaritat entre dues cadenes de car�cters. L'utilitzem a l'hora de generar regles autom�ticament, per tal de decidir quan hem de fusionar regles i com ho hem de fer.

\item{ConfigParser:}
M�dul que ens permet llegir el fitxer de configuraci� de l'aplicaci� al qual hi ha definits els diferents par�metres que guien l'aplicaci�.
\end{itemize}

\paragraph{}
Tamb� fem �s de les biblioteques seg�ent:
\begin{itemize}
\item{XGoogle:}
Es tracta d'una biblioteca for�a senzilla escrita per Peteris Krumins que ens facilita les cerques a Internet. L'objectiu del creador era oferir la possibilitat de cercar a \textit{Google} sense les limitacions de la API oficial, que nom�s deixa obtenir un m�xim de 32 resultats. Nosaltres l'hem ampliat per poder obtenir resultats de \textit{Google Scholar} i  posteriorment, tamb� per utilitzar les APIs oficials JSON de \textit{Google}, \textit{Bing} i \textit{Yahoo}. 

\item{BeautifulSoup:}
Es tracta d'un \textit{parser} d'HTML i XML desenvolupat per Leonard Richardson que genera arbres sint�ctics els quals podem consultar segons les etiquetes HTML, els atributs, o b� el text que contenen.

\item{\textit{Parser} de \BibTeX{}:}
Es tracta d'un \textit{parser} escrit per Raphael Ritz pel projecte Bibliograph.parsing. El fem servir per obtenir els diferents camps de les refer�ncies a l'hora d'importar-les a l'aplicaci� i a l'extreure-les mitjan�ant \textit{reference wrappers} (veure \ref{refextraction:reference-wrappers}) per tal de poder-les validar.

\item{\textit{SQLAlchemy}:} 
�s un ORM per \textit{Python} que falicita el treball amb la base de dades. A m�s, ofereix una capa d'abstracci� que permet utilitzar la base de dades de forma independent a la tecnologia. Per exemple, nosaltres fem servir \textit{SQLite}, per� en teoria fer un canvi del sistema gestor de la base de dades sense haver de tocar res m�s.

\item{PyQt:}
Es tracta d'una biblioteca en \textit{Python} que embolcalla el framework per la creaci� d'interf�cies gr�fiques \textit{Qt}.
\end{itemize}
