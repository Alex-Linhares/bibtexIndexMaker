\chapter{Generaci� de \textit{Wrappers}}
\label{chapter:wrapperinduction}
Els \textit{field wrappers} 


%%%% GENERACI� AUTO. DE REGLES %%%%
\section{Generaci� autom�tica de regles}
Com ja hem comentat al cap�tol anterior, els \textit{field wrappers} estan formats per una llista de regles que s'han d'aplicar en un ordre concret per tal de poder extreure el resultat. Per nosaltres una regla est� definida per un patr� i un procediment per aplicar-lo. En tindrem de diferents tipus segons les dades d'entrada i sortida que reben i retornen. Per poder-les concatenar, cal que la sortida d'una regla sigui v�lida per l'entrada de la seg�ent.

\paragraph{}
Donat que a nosaltres ens interessa extreure informaci� de documents HTML, les regles que com a m�nim necessitem s�n: una per localitzar la informaci� i una altra per extreure el valor.

\paragraph{}
Per generar les regles �s necessari tenir com a m�nim dos exemples de refer�ncies que continguin el camp que volem obtenir emmagatzemats a la base de dades de l'aplicaci�. Aquestes refer�ncies poden haver estat importades d'un fitxer \texttt{.bib} o b� extretes anteriorment utilitzant \textit{wrappers} que amb el temps han deixat de funcionar.
Tots els exemples hauran de tenir associada una URL que apunti a una p�gina que cont� la informaci� de la refer�ncia. Com hem comentat a la secci� \ref{chapger:refextraction:storage}, en el cas de les refer�ncies extretes autom�ticament, aquesta adre�a queda emmagatzemada durant l'extracci�.


\paragraph{}
El primer pas a realitzar amb els exemples �s comprovar que encara s�n v�lids. Per fer-ho, obtindrem d'Internet la p�gina 

\subsection{Ruta d'un element HTML}
El patr� d'aquest tipus de regles est� format per una llista de triples
Aquest tipus de regles defineixen la ruta per trobar un tros d'informaci� dins del document HTML que reben com a entrada.

\subsection{Expressi� regular}
xx

\subsection{Regles multi valor}
Alguns camps com ara els autors o editors tenen m�ltiples valors. Per poder-los extreure per separat, hem creat variants de les regles anteriors. Assumim que els m�ltiples valors corresponents al mateix cam copmliran una de les dues condicions seg�ents:
\begin{itemize}
\item{}
Es troben en elements HTML diferents, per� s�n germans o cosins. Aquesta condici� permet extreure valors quan es troben tant en llistes com en taules:

\begin{center}
\begin{minipage}{0.45\linewidth}
\begin{lstlisting}[language=HTML, title={Germans}, nolol=true]
<ul>
  <li>
    Liu Jing
  </li>
  <li>
  Li Jiandong
  </li>
  <li>
    Chen Yanhui
  </li>
</ul>
\end{lstlisting} 
\end{minipage}
\begin{minipage}{0.53\linewidth}
\begin{lstlisting}[language=HTML, title={Cosins}, nolol=true]
<table>
  <tr>
    <td>Autors:</td>
    <td>Liu Jing</td>
  </tr>
  
  <tr>
    <td></td>
    <td>Li Jiandong</td>
  </tr>
</table>
\end{lstlisting}  
\end{minipage}
\end{center}

\item{}
Es troben dins el mateix HTML amb un o m�s separadors entre ells. Per exemple, els tres autors seg�ents estan separats per les cadenes ``, '' i `` and ''.
\begin{center}
\begin{lstlisting}[language=HTML, caption={Exemple m�ltiples valors}, label=listing:exampleSiblingElements]
<td>Liu Jing, Li Jiandong and Chen Yanhui</td>
\end{lstlisting}
\end{center}

\end{itemize}

\section{Avaluaci� dels \textit{wrappers}}
Una vegada hem generat el conjunt dels \textit{wrappers} possibles per a un conjunt de documents, cal que avaluem quins d'ells funcionen millor. Utilitzem un sistema de vots positius i negatius i en calculem la mitjana amb la seg�ent f�rmula:

\begin{equation*}
    score = \frac{vots\; positius}{vots\; totals}
\end{equation*}

