\chapter{An�lisi de resultats}
\label{chapter:results}
En aquest cap�tol es mostren les principals proves realitzades per cadascuna de les tres parts del sistema que hem descrit als cap�tols anteriors


\section{Cerca de refer�ncies}
En primer lloc provarem com de b� ho fa el sistema a l'hora de cercar p�gines a Internet que continguin informaci� sobre un article concret. Els tests que hem dut a terme consisteixen en:
\begin{enumerate}
\item{Obtenir una s�rie de consultes d'un llistat de documents PDF}
\item{Cercar cadascuna de les consultes amb: \textit{Google}, \textit{Bing} i \textit{Yahoo}}
\item{Per cadascun dels resultats obtinguts, analitzem si �s bo o no}
\item{Comptabilitzem el n�mero de consultes que han fet falta per obtenir el primer \textit{bon} resultat}
\end{enumerate}

Sobre els passos anteriors, hem de notar que per tal classificar els resultats en bons i dolents nomes comprovem si part de la informaci� que volem es troba dins de la p�gina resultant. Aquesta no �s una soluci� perfecta, per� ens permet fer una aproximaci� sobre la quantitat de fitxers pels quals en podem trobar la refer�ncia.

\paragraph{}
Una altra q�esti� sobre la implementaci� dels tests, �s que els resultats obtinguts se solen repetir entre consultes del mateix article. Per estalviar temps i evitar fer moltes peticions seguides als mateixos servidors (que podrien resultar en un bloqueig), deixem uns segons entre petici� i petici� i emmagatzemem cada resultat de manera que nom�s l'haguem de demanar una sola vegada.
\paragraph{}
Tambe hem de comentar que en molts casos, alguns dels resultats corresponen al mateix fitxer PDF del qual estem buscant informaci� i que, per tant, els hem d'ometre.


\section{Extracci� de refer�ncies}


\section{Inducci� de \textit{wrappers}}





