\documentclass[a4paper,oneside]{article}
\usepackage{geometry}
\usepackage[latin1]{inputenc}
\usepackage[catalan]{babel}
\usepackage{amsfonts}
\usepackage{graphicx}
\usepackage{listings}
\usepackage{url} 

\usepackage[pdfauthor={Ramon Xuriguera Albareda},%
		pdfsubject={BibTeX Bibliography Index Maker},%
		pdftitle={BibTeX Bibliography Index Maker: Meeting notes},%
		pdftex]{hyperref}

\title{BibTeX Bibliography Index Maker: Meeting Notes}
\author{Ramon Xuriguera}
\date{03-02-2010}

\setlength{\parindent}{0in}

\begin{document}
\maketitle

\section{Estimaci� temporal per tasques/m�duls}
Cada m�dul tindr�, com a m�nim, les subtasques seg�ents: disseny, tests, codi i documentaci�. Els seg�ents m�duls s�n independents entre ells i per tant, es poden desenvolupar en qualsevol ordre o en paral�lel.
\begin{itemize}
\item{M�dul IR: } Recuperaci� de la informaci�, cerques utilitzant Google Scholar.\\
Punts a considerar:
    \begin{itemize}
        \item{Google retorna els diferents tipus de resultats segons la informaci� de la qual disposa.}
    \end{itemize}
Temps estimat: \textbf{7d}

\item{M�dul PDF: } Extracci� de text de PDFs.\\
Punts a considerar: 
    \begin{itemize}
        \item{Postproc�s del text extret dels PDFs.}
    \end{itemize}
Temps estimat: \textbf{10d}

\item{M�dul IE: } Extracci� de la informaci� de les p�gines bases de dades d'articles.\\
Punts a considerar:
    \begin{itemize}
        \item{Coficiaci� de wrappers per cada possible p�gina a tenir en compte}
        \item{Decisi� del wrapper a utilitzar (nom�s URL?)}
        \item{Decisi� del tipus de document (article/proceedings)}
    \end{itemize}
Temps estimat: \textbf{15d}

\item{M�dul Bibtex: } Generaci� de refer�ncies BibTex.\\
Temps estimat: \textbf{8d}
\end{itemize}


Aquest m�dul dep�n de tots els anteriors:
\begin{itemize}
\item{M�dul principal: } Encarregat d'establir el flux de la informaci�.\\
Punts a considerar:
    \begin{itemize}
        \item{Avaluaci� de la \textit{predicci�}}
        \item{Depenent del temps d'execuci� necessari per al conjunt de tasques anteriors degut a esperes per obtenir informaci� d'Internet pot ser necessari fer multithreading}
    \end{itemize}
Temps estimat: \textbf{6d}
\end{itemize}


Els seg�ents punts depenen del m�dul principal, per� es poden desenvolupar en paral�lel o en qualsevol ordre:
\begin{itemize}
\item{CLI: } Interf�cie de comandes.\\
Punts a considerar:
    \begin{itemize}
        \item{Integraci� amb el sistema operatiu. Per exemple, text d'ajuda al fer un \texttt{man}}
    \end{itemize}
Temps estimat: \textbf{4d}

\item{Plug-in JabRef: }\\
Punts a considerar:
    \begin{itemize}
        \item{S'utilitza una tecnologia diferent que en la resta del projecte.}
        \item{Interf�cie gr�fica}
    \end{itemize}
Temps estimat: \textbf{8d}
\end{itemize}


Finalment:
\begin{itemize}
\item{Distribuci�: }\\
Punts a considerar:
    \begin{itemize}
        \item{Empaquetar amb fitxers .egg i .jar respectivament.}
    \end{itemize}
Temps estimat: \textbf{3d}
\end{itemize}


Amb aquestes estimacions i tenint en compte els caps de setmana i setmana santa, el projecte estaria llest a principis de maig. Faltaria la mem�ria.

\end{document}
