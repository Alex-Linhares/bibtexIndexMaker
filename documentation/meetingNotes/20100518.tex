\documentclass[a4paper,oneside]{article}
\usepackage{geometry}
\usepackage{doc}
\usepackage[latin1]{inputenc}
\usepackage[catalan]{babel}
\usepackage{amsfonts}
\usepackage{graphicx}
\usepackage{listings}
\usepackage{fancyvrb}
\usepackage{url} 
\usepackage{color}
\usepackage{lscape}

\usepackage[pdfauthor={Ramon Xuriguera Albareda},%
		pdfsubject={BibTeX Bibliography Index Maker},%
		pdftitle={BibTeX Bibliography Index Maker: Meeting notes},%
		pdftex]{hyperref}

\lstset{%
    numbers=none,               %
    breaklines=true,            %
    fancyvrb=false,             %
    tabsize=2,                  % sets default tabsize to 2 spaces
    captionpos=b,               % sets the caption-position to bottom
    frame=single,
    xleftmargin=3em,
    xrightmargin=3em,
    backgroundcolor = \color{lightgrey}
}        


\title{\BibTeX{} Bibliography Index Maker: Meeting Notes}
\author{Ramon Xuriguera}
\date{18-05-2010}

\setlength{\parindent}{0in}
\definecolor{lightgrey}{gray}{0.85}

\begin{document}
\maketitle

\section{Proves}
Proves que podem realitzar:
\begin{itemize}
\item{Generaci� de Wrappers}
\begin{itemize}
\item{Comparar la qualitat dels \textit{wrappers} generats segons el n�mero d'exemples a partir dels quals s'ha obtingut.}\\
Per un parell de p�gines web mirar com varien els wrappers generats segons el n�mero de refer�ncies a partir dels quals s'obtenen. Provar amb 2, 3, 4, 5...
\item{Comparar el n�mero de camps d'informaci� que ofereix una p�gina respecte els wrappers que s'han pogut generar correctament (o gaireb�)}
\end{itemize}


\item{Cerca de refer�ncies}
Avaluar els resultats obtinguts al cercar refer�ncies segons els diferents par�metres:
\begin{itemize}
\item{M�nim/M�xim de paraules de la consulta}
\item{Consultes inicials que se salten per tal de descartar resultats on nom�s hi ha el t�tol}
\item{M�xim nombre de consultes que a provar}
\item{N�mero de resultats que fan que es consideri la cerca massa poc restrictiva (Too-many-results)}
\end{itemize}
Anar canviant els valors dels par�metres i comparar el n�mero de resultats i la qualitat segons cada cercador (Scholar, Google, Bing i Yahoo)


\item{Extracci� de refer�ncies}
\begin{itemize}
\item{Establir uns quants directoris amb PDFs d'una p�gina en concret}
\item{Limitar resultats de la web a la p�gina a la qual pertanyen els documents}
\item{Extreure les refer�ncies}
\item{Comptabilitzar refer�ncies encerts i comparar entre les diferents p�gines}
\end{itemize}

\end{itemize}

\section{Gui� Presentaci�}
\begin{itemize}
\item{}
Comen�ar amb BD buida
\item{}
Importar unes quantes refer�ncies des d'un fitxer \BibTeX
\item{}
Generar wrappers a partir d'aquestes refer�ncies. Fer-ho per un parell o tres de p�gines web.
\item{}
Escollir un directori amb uns quants PDF i extreure les refer�ncies.
\item{}
Mostrar les refer�ncies obtingudes i modificar errors (resultats bons per� no perfectes)
\item{}
Exportar les refer�ncies a un fitxer 
\end{itemize}

\end{document}


