\documentclass[a4paper,oneside]{article}
\usepackage{geometry}
\usepackage{doc}
\usepackage[latin1]{inputenc}
\usepackage[catalan]{babel}
\usepackage{amsfonts}
\usepackage{graphicx}
\usepackage{listings}
\usepackage{url} 

\usepackage[pdfauthor={Ramon Xuriguera Albareda},%
		pdfsubject={BibTeX Bibliography Index Maker},%
		pdftitle={BibTeX Bibliography Index Maker: Meeting notes},%
		pdftex]{hyperref}

\title{BibTeX Bibliography Index Maker: Meeting Notes}
\author{Ramon Xuriguera}
\date{24-02-2010}

\setlength{\parindent}{0in}

\begin{document}
\maketitle

\section{Multithreading}
He estudiat la possibilitat d'utilitzar diferents threads i sembla una opci� que pot ajudar a millorar els temps d'execuci� pels casos en que el n�mero de fitxers pels quals hem d'obtenir refer�ncies �s gran. Comparaci� del temps per obtenir m�ltiples p�gines aleat�ries d'Internet de forma seq�encial i paral�lela:

    \begin{center}
    \begin{tabular}{|r|r|r|r|r|r|r|r|}
        \hline
        \multicolumn{2}{|c|}{2 p�gines} & \multicolumn{2}{|c|}{5 p�gines} & \multicolumn{2}{|c|}{10 p�gines} & \multicolumn{2}{|c|}{20 p�gines} \\
        \hline
        Seq. & 5 Threads          & Seq. & 5 Threads          & Seq. & 5 Threads           & Seq. & 5 Threads \\
        \hline
        \hline
        0.9010 & 0.5481 & 2.1830 & 0.6612 & 4.3153 & 1.5914 & 7.9295 & 2.5949 \\
        0.7467 & 0.3795 & 2.1558 & 0.7441 & 4.3186 & 1.2311 & 8.5483 & 2.1958 \\ 
        0.7678 & 0.5641 & 2.0645 & 0.5383 & 9.2930 & 1.4415 & 8.7202 & 2.5749 \\
        0.7421 & 0.3876 & 2.0684 & 0.8551 & 4.9859 & 1.5294 & 8.4732 & 2.2841 \\
        0.9674 & 0.5477 & 2.1510 & 0.8550 & 5.3600 & 1.3116 & 9.2901 & 2.2257 \\
        \hline
        \multicolumn{8}{|l|}{Mitjana:} \\
        \hline
        0.8250 & 0.4854 & 2.1246 & 0.7307 & 5.6546 & 1.4210 & 8.5923 & 2.3751 \\
        \hline
        \multicolumn{8}{|l|}{Guany:} \\
        \hline
        \multicolumn{2}{|c|}{\textbf{-0.34s}} &  \multicolumn{2}{|c|}{\textbf{-1.39s}} &  \multicolumn{2}{|c|}{\textbf{-4.23s}} &  \multicolumn{2}{|c|}{\textbf{-6.22s}} \\
        \hline
    \end{tabular}
    \end{center}
Els temps de la taula tenen en compte el temps necessari per crear i finalitzar els diferents fils d'execuci�. Els resultats s'han obtingut al fer consultes aleat�ries a Google. El temps necessari per obtenir altres p�gines seran m�s alts. Hem utilitzat consultes aleat�ries per evitar l'efecte dels proxys i cach�s.
\paragraph{}
Algunes idees:
\begin{itemize}
    \item{Utilitzar m�ltiples fils o no depenent del n�mero de fitxers pels quals hem d'obtenir la refer�ncia}
    \item{Tenir un pool de fils d'execuci� que consumeixen fitxers d'una cua d'entrada i emmagatzemen els resultats en una cua de sortida. Els fils es reutilitzen.}
    \item{Ajustar el n�mero m�xim de fils segons la velocitat de la xarxa}
\end{itemize}

\section{SQLite}
Segons la documentaci� �s perfectament adequat per aplicacions que tenen un n�mero inferior a 100K hits diaris, un n�mero molt superior del que nosaltres necessitem.

\section{Wrapper induction}
Una idea:
\begin{enumerate}
    \item{A l'obtenir una refer�ncia emmagatzemar les dades a BD}\\
    Dades que podr�em guardar: ruta del pdf al sistema, url a partir de la qual hem obtingut la refer�ncia, camps principals
    \item{Per generar un wrapper, identifiquem on es troba cada pe�a d'informaci� dins de la p�gina html}\\
    Podem comprovar qu� falla abans de tornar a generar tot el wrapper.
    \item{Generem un fitxer de configuraci� del wrapper}
    \item{Configuration-driven wrappers:}\\
    Generalitzem el codi dels wrappers per tal que tinguin un comportament diferent per configuracions diferents.
\end{enumerate}

A considerar:
\begin{itemize}
    \item{La soluci� nom�s funciona per \textit{field wrappers}}
    \item{Si una p�gina passa a canviar el format de les seves urls (poc habitual), no es podr� regenerar el wrapper:}\\
    Podem tornar a cercar-la amb google i actualitzar la url.
\end{itemize}
\end{document}


