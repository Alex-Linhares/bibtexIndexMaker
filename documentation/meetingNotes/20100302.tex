\documentclass[a4paper,oneside]{article}
\usepackage{geometry}
\usepackage{doc}
\usepackage[latin1]{inputenc}
\usepackage[catalan]{babel}
\usepackage{amsfonts}
\usepackage{graphicx}
\usepackage{listings}
\usepackage{url} 

\usepackage[pdfauthor={Ramon Xuriguera Albareda},%
		pdfsubject={BibTeX Bibliography Index Maker},%
		pdftitle={BibTeX Bibliography Index Maker: Meeting notes},%
		pdftex]{hyperref}

\title{BibTeX Bibliography Index Maker: Meeting Notes}
\author{Ramon Xuriguera}
\date{02-03-2010}

\setlength{\parindent}{0in}

\begin{document}
\maketitle

\section{SpringerLink}
Google facilita l'enlla� al PDF d'SpringerLink, no a la p�gina amb la refer�ncia. Si s'intenta accedir-hi des del navegador, se'ns redirecciona directament a la p�gina correcta, per� si volem obtenir el contingut amb el nostre \texttt{Browser} de Python, hi ha dues redireccions que ens acaben portant a la p�gina principal. Per exemple:
\begin{itemize}
\item{}
Amb l'enlla�: \url{http://www.springerlink.com/index/CGBHLDY69PVGWL6D.pdf}, obtenim un \texttt{302 Found} que ens porta a \url{http://www.springerlink.com/link.asp?id=cgbhldy69pvgwl6d}.
\item{}
Seguint aquesta adre�a, obtenim un altre \texttt{302 Found} que ens porta a \url{http://www.springerlink.com/default.mpx}. 
\end{itemize}
Aix� �s degut a que les cookies no estan habilitadies i no es pot emmagatzemar l'identificador de la sessi� ASP d'SpringerLink.
\\
\\
La soluci� ha estat afegir un \textit{handler} per a poder utilitzar cookies: \texttt{handlers = [PoolHTTPHandler, urllib2.HTTPCookieProcessor]}


\section{Bloqueig de Google}
Si s'executen moltes (en realitat, no moltes) consultes a la vegada Google passa a bloquejar el servei i tarda unes hores a desbloquejar-se. 

Possibles solucions:
\begin{itemize}
    \item{}
    Utilitzar Tor o similar per anonimitzar el tr�fic. La velocitat de transfer�ncia �s molt menor a l'habitual. Hi ha timeouts.
    \item{}
    Fer que l'usuari vagi introduint \textit{captchas} de tant en tant. He estat treballant en un script per poder introduir a m� el captcha i desbloquejar-ho, per� encara no funciona del tot.
\end{itemize}

\section{Versions del mateix article}
M'he trobat amb articles com ara el que porta per t�tol \textit{Logical Implication and Causal Dependency} que al buscar-lo per Internet nom�s hi ha una versi� (se suposa que m�s nova) i que t� un t�tol diferent. Es pot trobar a \url{http://www.springerlink.com/content/p0086762337rq30t/}.
\\
\\
Un altre exemple: l'article amb nom \textit{Prince: An Algorith for...} es pot trobar a \url{http://www.springerlink.com/content/h8mja4yu8279mjq1/} amb el nom \textit{TEX: A New Informative Generic Base of Association Rules}.
Qu� cal fer en aquests casos? Com ho podem detectar?


\end{document}


